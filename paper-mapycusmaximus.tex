% !TeX root = RJwrapper.tex
\title{Focus-Glue-Context Fisheye Transformations for Spatial Visualization}


\author{by Thanh Cuong Nguyen, Michael Lydeamore, and Dianne Cook}

\maketitle

\abstract{%
Fisheye views magnify local detail while preserving context, yet projection-aware, scriptable tools for R spatial analysis remain limited. mapycusmaximus introduces a Focus--Glue--Context (FGC) fisheye transform for numeric coordinates and sf geometries. Acting radially around a chosen center, the transform defines a magnified focus (r\_in), a smooth transitional glue zone (r\_out), and a fixed exterior. Distances expand or compress via a zoom factor and a power-law squeeze, with an optional angular twist that enhances continuity. The method is projection-conscious: lon/lat inputs are reprojected to suitable CRSs (e.g., GDA2020/MGA55), normalized for stable parameter control, and restored afterward. A geometry-safe engine (st\_transform\_custom) supports all feature types, maintaining ring closure and metadata. The high-level sf\_fisheye() integrates with tidyverse, ggplot2, and Shiny, with built-in datasets and tests ensuring reproducibility. By coupling coherent radial warps with tidy, CRS-aware workflows, mapycusmaximus enables spatial exploration that emphasizes local structure without losing global context.
}

\section{Introduction}\label{introduction}

Maps that reveal fine local structure without losing broader context face a persistent challenge: zooming in hides regional patterns, while small-scale views suppress local detail. Traditional solutions---insets, multi-panel displays, aggressive generalization---break spatial continuity and increase cognitive load \citep{cockburn2008}. What if we could smoothly magnify a metropolitan core \emph{while keeping it embedded} in its state-level context?

This package implements a Focus--Glue--Context (FGC) fisheye transformation that continuously warps geographic space. The transformation magnifies a chosen focus region, compresses surrounding areas into a transitional glue zone, and maintains stability in the outer context. The approach operates directly on vector geometry coordinates, preserves topology, and supports reproducible, pipeline‑oriented cartography within the R sf and ggplot2 ecosystem. An optional glue‑zone twist (the \texttt{revolution} parameter) can gently rotate features to aid continuity; in this paper's figures we set \texttt{revolution\ =\ 0}.

The intellectual lineage of focus+context visualization traces back to \citet{furnas1986}'s \emph{degree-of-interest} function, which introduced a formal method to rank information elements by combining intrinsic importance with distance from the user's focus. In this model, items with low DOI are deemphasized or hidden, enabling emphasis on salient regions without losing global structure. \citet{sarkar1992} and \citet{sarkar1994} extended this to geometric distortion, demonstrating smooth magnification transitions for graph visualization. Subsequent innovations explored diverse lenses: hyperbolic geometry for hierarchies \citep{lamping1995}, distortion-view frameworks \citep{carpendale2001}, and ``magic lens'' overlays \citep{bier1993}. By 2008, \citet{cockburn2008}'s comprehensive review synthesized two decades of research across overview+detail, zooming, and focus+context paradigms.

In cartography, the need for nonlinear magnification emerged independently. \citet{snyder1987} developed ``magnifying-glass'' azimuthal projections with variable radial scales---mathematical foundations. \citet{harrie2002} created variable-scale functions for mobile devices where user position appears large-scale against small-scale surroundings. An influential contribution came from \citet{yamamoto2009} and \citet{yamamoto2012}: their \textbf{Focus+Glue+Context model} introduced an intermediate ``glue'' region that absorbs distortion, preventing the excessively warped roads and boundaries that plagued earlier fisheye maps. This three-zone architecture proved particularly effective for pedestrian navigation and mobile web services.

Within R's spatial ecosystem, sf \citep{RJ-2018-009} provides robust vector handling and CRS transformations, while ggplot2 \citep{wickham2016} offers declarative visualization grammar. Yet a gap remained: existing tools addressed \emph{related} distortion needs but not continuous geometric fisheye lenses. This package fills that niche by formalizing an sf-native FGC radial model with controllable zone parameters, optional angular effects, automatic normalization, and safe geometry handling across points, lines, and polygons.

\section{Background}\label{background}

Before examining the mechanics of fisheye transformations, it is important to review how R's spatial ecosystem currently addresses the detail-versus-context tradeoff. This context clarifies why existing solutions, though valuable, do not fully address the need for continuous lens-based warping.

\subsubsection{Cartograms: Thematic distortion}\label{cartograms-thematic-distortion}

The cartogram family \citep{gastner2004} intentionally distorts geographic areas to encode variables---population density reshapes regions so area becomes proportional to demographic weight (see Figure \ref{fig:plot-cart}).

\begin{figure}
\centering
\pandocbounded{\includegraphics[keepaspectratio]{paper-mapycusmaximus_files/figure-latex/plot-cart-1.pdf}}
\caption{\label{fig:plot-cart}Cartogram example: thematic distortion changes shapes and sizes to encode population.}
\end{figure}

This approach fundamentally differs from focus+context methods. Cartograms substitute spatial accuracy for data encoding, often severely disrupting shapes and adjacencies. For example, a population cartogram enlarges Monash while shrinking La Trobe, prioritizing thematic insight over geographic fidelity. The reader can still somehow find the shape of Victoria familiars, as the cartogram preserves relative positions and topology, however, it shape and size are distorted. In contrast, the FGC fisheye transformation preserves relative positions and topology while magnifying a user-selected spatial region rather than a data-driven variable. The use cases are distinct: cartograms address the dominance of a variable in space, whereas fisheye lenses facilitate exploration of local detail within a broader geographic context.

\subsubsection{Hexagon tile maps: Discrete abstraction}\label{hexagon-tile-maps-discrete-abstraction}

Packages like \texttt{sugarbag} replace irregular polygons with regular hexagonal or square tiles, each representing an administrative unit.

\begin{figure}
\centering
\pandocbounded{\includegraphics[keepaspectratio]{paper-mapycusmaximus_files/figure-latex/sugarbag-plot-1.pdf}}
\caption{\label{fig:sugarbag-plot}Sugarbag example: thematic distortion changes shapes and sizes to encode population.}
\end{figure}

As seen in Figure \ref{fig:sugarbag-plot}, tile maps \emph{abstract away} precise geography entirely, treating space as a topology-preserving tessellation where ``neighbors touch'' matters more than accurate boundaries. Tile maps excel at avoiding size bias (Mildura gets equal visual weight to Yarra) and creating aesthetic, clutter-free layouts. However, they abandon continuous spatial relationships: you cannot identify precise locations, measure distances, or overlay point data meaningfully. In the example above, the sugarbag approach was overlay on top of the orignal geography. Hexbin aggregation for point data (via \texttt{ggplot2::geom\_hex()}) serves a different purpose---density estimation---rather than focus+context navigation.

\subsubsection{Multi-panel approaches: Spatial separation}\label{multi-panel-approaches-spatial-separation}

Tools like \texttt{cowplot::ggdraw()}\citep{cowplot} create side-by-side views: one panel shows overview, another shows zoomed detail (Figure \ref{fig:cow-plot-plot}).

\begin{figure}
\centering
\pandocbounded{\includegraphics[keepaspectratio]{paper-mapycusmaximus_files/figure-latex/cow-plot-plot-1.pdf}}
\caption{\label{fig:cow-plot-plot}Overview with inset: separates focus from context into distinct panels.}
\end{figure}

These are effective for static reports but require viewers to mentally integrate separate views, and they don't preserve the \emph{embedded} relationship between focus and context within a single continuous geography. Futhermore, if you introduce one or more elements into the plot like filling value equal to a variable, the audience will have a hard time identify the zoomed detail.

\subsubsection{Why FGC fisheye offers something distinct}\label{why-fgc-fisheye-offers-something-distinct}

None of these approaches provide \emph{continuous geometric magnification within a single, topology-preserving map}. Cartograms distort for data, not user-chosen focus. Tile maps abstract away geography. Multi-panel tools spatially separate context. The fisheye lens keeps everything in one frame---roads bend smoothly, metropolitan detail enlarges, but you still see how the city sits within its state. It's a geometric \emph{warp} rather than a data-driven \emph{substitution} or panel-based \emph{separation}. This matters for use cases like: examining hospital networks in Melbourne while maintaining Victorian context, exploring census tracts in a metro core without losing county boundaries, or analyzing transit lines with their regional hinterland visible.

With this landscape established, we now turn to the technical implementation: how does the FGC transformation actually work, and how does this package make it accessible within R's spatial workflows?

\section{Implementation}\label{implementation}

\begin{figure}
\centering
\pandocbounded{\includegraphics[keepaspectratio]{paper-mapycusmaximus_files/figure-latex/fgc-zones-1.pdf}}
\caption{\label{fig:fgc-zones}The three zones of an FGC transformation. Points inside the focus (red) expand radially; points in the glue (blue) compress toward the focus boundary; context points (gold) remain fixed.}
\end{figure}

\begin{verbatim}
# Inspect diagnostics returned by fisheye_fgc()
head(transform_df[, c("x_new", "y_new", "zone", "r_orig", "r_new")])
\end{verbatim}

\begin{verbatim}
#> # A tibble: 6 x 5
#>   x_new y_new zone    r_orig r_new
#>   <dbl> <dbl> <chr>    <dbl> <dbl>
#> 1  -1      -1 context   1.41  1.41
#> 2  -0.9    -1 context   1.35  1.35
#> 3  -0.8    -1 context   1.28  1.28
#> 4  -0.7    -1 context   1.22  1.22
#> 5  -0.6    -1 context   1.17  1.17
#> 6  -0.5    -1 context   1.12  1.12
\end{verbatim}

\begin{verbatim}
table(transform_df$zone)
\end{verbatim}

\begin{verbatim}
#> 
#> context   focus    glue 
#>     362      37      42
\end{verbatim}

Consider a point \(P = (x, y)\) in a projected coordinate system. The analyst chooses a center \(C =
(c_x, c_y)\) and two radii: \(r_{\text{in}}\) delineating the focus region and \(r_{\text{out}}\)
marking the glue boundary. Points inside the focus magnify, points between the radii focus on the center and then compress according to a smooth curve, and points outside remain unchanged. This radial scheme keeps angular
coordinates intact, thereby preserving bearings and relative direction.

\subsection{Algorithm}\label{algorithm}

Let \((r, \theta)\) denote the polar form of point \(P = (x, y)\) relative to center \(C = (c_x, c_y)\). The transformation defines a new radius \(r'\) via a piecewise function:

\begin{equation} 
 r' = 
 \begin{cases}
   \min\left( z \cdot r, r_{\text{in}} \right) & \text{if } r \le r_{\text{in}}, \\
   r_{\text{in}} + (r_{\text{out}} - r_{\text{in}}) \cdot h(u; s) & \text{if } r_{\text{in}} < r \le r_{\text{out}}, \\
   r & \text{if } r > r_{\text{out}},
 \end{cases}
\end{equation}

where \(z > 1\) is the zoom factor within the focus, \(s \in (0, 1]\) controls glue compression, and
\(u = { (r - r_{\text{in}}) }/{ (r_{\text{out}} - r_{\text{in}}) }\) normalises the glue radius to
\([0,1]\). The function \(h(u; s)\) is chosen so that \(h(0; s) = 0\), \(h(1; s) = 1\), and both the
first derivatives and the radii match at the boundaries. We adopt a symmetric power curve:

\begin{equation} 
 h(u; s) = 
 \begin{cases}
   \tfrac{1}{2} \cdot u^{1/s} & \text{if } 0 \le u \le 0.5, \\
   1 - \tfrac{1}{2} \cdot (1 - u)^{1/s} & \text{if } 0.5 < u \le 1,
 \end{cases}
\end{equation}

which compresses radii near both boundaries and emphasises the mid-glue region. Analysts seeking
outward compression can choose alternative methods (e.g., the \texttt{"outward"} mode) that bias the curve
towards \(r_{\text{out}}\). The demonstration on how original and transformed radius can be seen at the Figure \ref{fig:radial-curve}.
The transform optionally introduces rotation within the glue zone to accentuate the flow from detail
to context. Let \(\phi(u)\) denote the angular adjustment. We employ a bell-shaped profile:
\(\phi(u) = \rho \cdot 4u(1-u)\), where \(\rho\) is the revolution parameter (in radians). This
function peaks at the glue midpoint and vanishes at the boundaries, ensuring continuity.

\begin{figure}
\includegraphics[width=1.2\linewidth]{paper-mapycusmaximus_files/figure-latex/radial-curve-1} \caption{Radial mapping r→r' across focus, glue, and context.}\label{fig:radial-curve}
\end{figure}

\subsection{Integration with sf}\label{integration-with-sf}

Spatial datasets vary widely in CRS, extent, feature types, and schemas. mapycusmaximus follows a disciplined staged workflow where each step is explicit, auditable, and invariant to input type. The architecture separates numeric mapping, spatial orchestration, and geometry reconstruction, allowing the core transform to remain small and testable while sf-specific concerns are isolated in thin wrappers.

\subsubsection{Workflow and CRS handling}\label{workflow-and-crs-handling}

The pipeline proceeds: \textbf{sanitize input} → \textbf{select working CRS} → \textbf{normalize} → \textbf{warp} → \textbf{denormalize} → \textbf{restore original CRS}. Empty geometries are dropped and \texttt{sf::st\_zm()} enforces 2D coordinates.

\paragraph{CRS selection}\label{crs-selection}

If the layer is already in a projected CRS, that CRS is used. If it is geographic (lon/lat), the data are transformed to a sensible local projected CRS (e.g., UTM inferred from the centroid; for Victoria, GDA2020/MGA55 is typical). Distances are then in metres and parameters behave consistently. The original CRS is restored on return.

\paragraph{Normalization}\label{normalization}

\begin{figure}

{\centering \includegraphics[width=0.8\linewidth]{diagram} 

}

\caption{Diagram of the normalization step}\label{fig:norm-diagram-gg}
\end{figure}

A bounding box defines the normalizing scale. With \texttt{preserve\_aspect\ =\ TRUE}, a uniform scale \(s = \max(s_x, s_y)\) is applied; otherwise axes scale independently. Center resolution happens before normalization: \texttt{sf}/\texttt{sfc} centres reduce to a centroid then transform to the working CRS; numeric pairs with \texttt{center\_crs} are transformed; numeric pairs without CRS are interpreted heuristically; with \texttt{normalized\_center\ =\ TRUE}, pairs live in \([-1, 1]\) relative to the bbox midpoint. If no center is given, the bbox midpoint is used.

\subsubsection{Core transformation}\label{core-transformation}

At the heart of the package is \texttt{fisheye\_fgc()}, a vectorized function mapping an \(n \times 2\) coordinate matrix to a new \(n \times 2\) matrix via the FGC rule. Its contract is minimal: numeric arrays and scalar parameters defining center, radii, magnification, compression, method, and revolution. Internally it converts to polar form, applies the piecewise radial map with smooth boundary conditions, optionally perturbs angle via bell-shaped rotation, and converts back to Cartesian.

\pandocbounded{\includegraphics[keepaspectratio]{paper-mapycusmaximus_files/figure-latex/basic-example-1.pdf}}

It attaches diagnostic attributes (zone labels, original and new radii) consumed by plotting utilities but not affecting geometry reconstruction.

\begin{verbatim}
#>      x_new y_new
#> [1,]  -1.0    -1
#> [2,]  -0.9    -1
#> [3,]  -0.8    -1
#> [4,]  -0.7    -1
#> [5,]  -0.6    -1
#> [6,]  -0.5    -1
\end{verbatim}

\begin{verbatim}
#> [1] "dim"             "dimnames"        "zones"           "original_radius"
#> [5] "new_radius"
\end{verbatim}

Numeric stability at zone boundaries is ensured by clamping expansions in the focus so radii do not exceed \(r_{in}\), and using smooth power curves in the glue so derivatives match across boundaries. The radial mapping is vectorized and runs in linear time in the number of vertices as seen in the benchmark \ref{fig:bench-plot}.

\begin{figure}
\centering
\pandocbounded{\includegraphics[keepaspectratio]{paper-mapycusmaximus_files/figure-latex/bench-plot-1.pdf}}
\caption{\label{fig:bench-plot}Benchmark performance of fisheye\_fgc() and sf\_fisheye()}
\end{figure}

\subsubsection{Geometry reconstruction}\label{geometry-reconstruction}

At the top level is an all-in-one function \texttt{sf\_fisheye()}, which presents the user-facing interface while keeping the numeric core untouched. It validates input, selects working CRS, resolves center, constructs normalization closures, and invokes \texttt{st\_transform\_custom()} to rebuild geometries.

The geometry walker \texttt{st\_transform\_custom()} acts as a drop-in analogue to \texttt{sf::st\_transform()} but applies an arbitrary coordinate function. For each feature, it extracts coordinates via \texttt{sf::st\_coordinates()}, yielding a matrix with columns \((x, y, L1, L2, \dots)\) where L1 and L2 index polygon rings and multi-polygon parts. Geometries are split by type:

\begin{itemize}
\tightlist
\item
  \textbf{POINT}: direct warp
\item
  \textbf{LINESTRING}: warp each vertex, retain order\\
\item
  \textbf{POLYGON}: process each ring (identified by L1) independently
\item
  \textbf{MULTIPOLYGON}: nested by (L1, L2) combinations
\end{itemize}

After transformation, polygon rings are explicitly closed by forcing first and last vertices to equality: \((x_1', y_1') = (x_n', y_n')\). This prevents numerical drift when the warp changes ring curvature. Geometries are rebuilt using sf constructors (\texttt{st\_point()}, \texttt{st\_linestring()}, \texttt{st\_polygon()}, \texttt{st\_multipolygon()}), combined into an sfc with original CRS, and spliced back into an sf if appropriate. Attributes are preserved because only the geometry column is replaced.

Table \ref{tab:transformation-table} illustrates coordinate transformations across zones for a vertical transect, showing radial expansion in the focus, smooth compression in the glue, and identity mapping in the context.

\begin{table}

\caption{\label{tab:transformation-table}Coordinate transformation across zones for selected points}
\centering
\begin{tabular}[t]{r|r|r|r|l|r|r}
\hline
x & y & x\_new & y\_new & zone & r\_orig & r\_new\\
\hline
-1.0 & -1 & -1.000 & -1.000 & context & 1.414 & 1.414\\
\hline
-0.9 & -1 & -0.900 & -1.000 & context & 1.345 & 1.345\\
\hline
-0.8 & -1 & -0.800 & -1.000 & context & 1.281 & 1.281\\
\hline
-0.7 & -1 & -0.108 & -0.323 & focus & 0.316 & 0.340\\
\hline
-0.6 & -1 & 0.000 & -0.340 & focus & 0.300 & 0.340\\
\hline
-0.5 & -1 & 0.108 & -0.323 & focus & 0.316 & 0.340\\
\hline
-0.4 & -1 & 0.000 & -0.500 & glue & 0.500 & 0.500\\
\hline
-0.3 & -1 & -0.300 & -0.400 & glue & 0.500 & 0.500\\
\hline
-0.2 & -1 & -0.200 & -0.400 & glue & 0.447 & 0.448\\
\hline
\end{tabular}
\end{table}

\subsubsection{Design and extensibility}\label{design-and-extensibility}

Utilities in \texttt{utils.R} provide \texttt{create\_test\_grid()} for diagnostics, \texttt{classify\_zones()} for labeling, and \texttt{plot\_fisheye\_fgc()} for visualization. Dataset documentation in \texttt{data.R} accompanies example layers (vic, vic\_fish, conn\_fish) used in tests.

For multi-layer maps, the normal process is combine all the layers into a single sf object and apply \texttt{sf\_fisheye()}, then split the result later. One minialist example for this approach is show in the code block below.

\begin{verbatim}
# Multi-layer example
bind <- dplyr::bind_rows(
  object_1 |> dplyr::mutate(.layer="object_1"), 
  object_2 |> dplyr::mutate(.layer="object_2"))

bind_w <- sf_fisheye(
  bind, 
  center = melb, 
  r_in = 0.34, 
  r_out = 0.55, 
  zoom = 1.8, 
  squeeze = 0.35)

object_1_transformed   <- bind_w |> 
  dplyr::filter(.layer == "object_1") |> 
  dplyr::select(-.layer)

object_2_transformed   <- bind_w |> 
  dplyr::filter(.layer == "object_2") |> 
  dplyr::select(-.layer)
\end{verbatim}

The test suite mirrors the modular structure, covering boundary behaviour, zone labeling, CRS round‑trips, ring closure, and performance. Functions follow tidyverse‑oriented conventions (snake\_case parameters, small exported surface). Behaviour is validated by tests; we aim for stability across versions but do not promise guarantees.

\subsection{Parameters}\label{parameters}

The principal user interface is \texttt{sf\_fisheye()}, which accepts an \texttt{sf} or \texttt{sfc} object
and returns an object of the same top-level class whose geometry has been warped in a projection-
aware manner. For clarity, we group arguments into data/CRS handling, centre selection, and radial
warping, and we make explicit the invariants enforced by the implementation.

\textbf{Data and CRS.} The argument \texttt{sf\_obj} supplies the features to be transformed. Before any
calculation, empty geometries are removed and Z/M dimensions are dropped using \texttt{sf::st\_zm()}, so
that downstream computation operates on a strict \(n\times 2\) coordinate matrix. The optional
\texttt{target\_crs} sets the working projected CRS; if provided, the input is transformed via
\texttt{sf::st\_transform()} and the original CRS is restored on return. When \texttt{target\_crs\ =\ NULL}
and the input is geographic (lon/lat), a projected working CRS is chosen deterministically from the
layer's centroid: the default value is GDA2020 /; otherwise a
UTM zone is inferred by longitude and hemisphere. This choice ensures the fisheye operates in metric units with bounded distortion across the extent of interest. The \texttt{preserve\_aspect} flag governs normalisation: with \texttt{TRUE} (default) a uniform scale \(s = \max(s_x, s_y)\) is applied, where
\(s_x, s_y\) are bbox half-spans; with \texttt{FALSE}, independent scales are used per axis. Uniform scaling preserves circular symmetry of the focus and glue; per-axis scaling yields an ellipticalinterpretation that can be useful for long, narrow extents but should be used deliberately. Degenerate
cases (\(s_x = 0\) or \(s_y = 0\)) are handled by substituting a unit scale to avoid division by zero.

\textbf{Centre selection.} The lens centre may be specified in several forms. The preferred interface
is \texttt{center}, which takes precedence over legacy \texttt{cx}, \texttt{cy}. If \texttt{center} is a
numeric pair and \texttt{center\_crs} is provided (e.g., \texttt{"EPSG:4326"}), the point is transformed
into the working CRS. If \texttt{center\_crs} is omitted, a heuristic interprets pairs that lie within
\(|\text{lon}|\le 180\), \(|\text{lat}|\le 90\) as WGS84 and transforms them accordingly; otherwise
the values are assumed to be already in working-CRS map units. Any \texttt{sf}/\texttt{sfc} geometry may
be used as \texttt{center}; non-point centres are combined and reduced to a centroid and then
transformed to the working CRS, which is often convenient when the focal area is a polygon (e.g., a
CBD boundary) or a set of points (e.g., incident locations). Finally, when the argument \texttt{\{normalized\_center\ =\ TRUE\}}, \texttt{center} is interpreted as a pair in \([-1,1]\) relative to the bbox midpoint and the
chosen normalisation (uniform or per-axis). Normalised centres make parameter sets portable across
datasets of different extents and are a natural fit for parameter sweeps in reproducible pipelines.
If no centre is supplied, the bbox midpoint is used; this default is stable under reprojection.

\textbf{Radial warping.} The radii \texttt{r\_in} and \texttt{r\_out} define the focus and glue boundaries
in the normalised coordinate space and must satisfy \texttt{r\_out\ \textgreater{}\ r\_in}. The interpretation of these
radii depends on \texttt{preserve\_aspect}. With uniform scaling, a circle of radius \(r_{\text{in}}\)
in unit space corresponds to a circle of radius \(r_{\text{in}}\,s\) in map units; with per-axis
scaling, the corresponding shape is an axis-aligned ellipse with semi-axes \(r_{\text{in}}s_x\) and
\(r_{\text{in}}s_y\). Inside the focus, distances from the centre are multiplied by
\texttt{zoom\_factor}; to prevent overshoot, the implementation clamps \(r'\) so that points do not cross
the \(r_{\text{in}}\) boundary. Across the glue, \texttt{squeeze\_factor} in \((0,1]\) controls how strongly
intermediate radii compress: smaller values create tighter compression near the boundaries and a
more pronounced ``shoulder'' in the middle of the glue; larger values approach a linear transition. The
\texttt{method} selects the family of curves used in the glue. The default \texttt{"expand"} applies a
symmetrical power law that expands inward and outward halves of the glue to maintain visual balance
around the midpoint; \texttt{"outward"} biases the map towards \(r_{\text{out}}\), keeping the outer
boundary steadier and pushing more deformation into the inner portion of the glue. The optional
\texttt{revolution} parameter adds a bell-shaped angular twist inside the glue of magnitude
\(\rho\,4u(1-u)\), where \(u\) is the normalised glue radius. This rotation vanishes at both glue
boundaries and peaks at the midpoint, preserving continuity. Positive values rotate
counter-clockwise, negative values clockwise; values are specified in radians.

\textbf{Inter-parameter interactions and invariants.} The following constraints and behaviours are
enforced: \(r_{\text{out}} > r_{\text{in}} > 0\); \texttt{zoom\_factor} \(\ge 1\) (values close to one
yield gentle focus); \texttt{squeeze\_factor} in \((0,1]\) (\(=1\) approaches linear); and monotonicity of
the radial map so that ordering by distance from the centre is preserved. The choice of
\texttt{preserve\_aspect} affects the physical size of radii and thereby the impact of a given parameter
set on different datasets; using uniform scaling with a normalised centre yields the most portable
configurations. Twisting via \texttt{revolution} is confined to the glue; it does not change radii and
therefore does not affect the classification of points into zones. Because angles are modified only
in the glue, bearings inside the focus and in the context are preserved.

\textbf{Return value and side effects.} The function returns an object of the same top-level class as
its input (\texttt{sf} or \texttt{sfc}). For \texttt{sf} inputs, non-geometry columns are preserved
verbatim; only the geometry column is replaced. The original CRS is restored before return so that
downstream plotting and analysis code does not need to change. On malformed geometries, the
implementation emits a warning and returns an empty geometry of the appropriate family to preserve
row count and indices. For exploratory diagnostics, the low-level \texttt{fisheye\_fgc()} returns a
coordinate matrix with attributes \texttt{"zones"}, \texttt{"original\_radius"}, and \texttt{"new\_radius"};
these can be used to plot scale curves and verify parameter effects prior to applying the transform
to complex geometries.

\subsection{Common choices}\label{common-choices}

Although the parameter space is continuous, certain regimes recur in practice and can serve as
reliable starting points. We describe these regimes and articulate the trade-offs that motivate each
choice. The recommendations assume the default \texttt{preserve\_aspect\ =\ TRUE}; when per-axis scaling is
enabled, translate radii to semi-axes using the bbox half-spans.

\textbf{Quick start (synthetic grid).} Set \(r_{\text{in}}\) to 0.30--0.35 and \(r_{\text{out}}\) to
0.55--0.70. Pair this with \texttt{zoom\_factor} between 5 and 10 and \texttt{squeeze\_factor} near 0.35 for a
balanced focus that still shows context. Stick with \texttt{method\ =\ "expand"} and \texttt{revolution\ =\ 0} unless
you explicitly need outer rigidity or a twist.

In this simple example, we demo the effect of zoom factor 1.5 and 2 on a synthetic grid to demonstrate how the point movement was effected by the zoom factor.

\begin{figure}
\includegraphics[width=0.8\linewidth]{paper-mapycusmaximus_files/figure-latex/zoom-factor-plot-1} \caption{Zoom factors for balanced metropolitan focus within a state.}\label{fig:zoom-factor-plot}
\end{figure}

However, as demonstrated below, the revolution effect might make the distortion of the linestring object more obvious, comparing to just only the zoom factor effect.

\begin{figure}
\includegraphics[width=0.8\linewidth]{paper-mapycusmaximus_files/figure-latex/unnamed-chunk-1-1} \caption{Revolution effect}\label{fig:unnamed-chunk-1}
\end{figure}

For polygon object, we can see the same distortion effect as the linestring.

\begin{figure}
\includegraphics[width=0.8\linewidth]{paper-mapycusmaximus_files/figure-latex/poly-plot-1} \caption{Revolution effect with polygon}\label{fig:poly-plot}
\end{figure}

As we can see, the revolution effect create a vortex or zoom wheel effect into the focus zone, which may be useful in some cases. For manuscripts and dashboards, prefer \texttt{revolution\ =\ 0}.

\pandocbounded{\includegraphics[keepaspectratio]{paper-mapycusmaximus_files/figure-latex/unnamed-chunk-2-1.pdf}}

Similarly, start with \texttt{"expand"} and adopt \texttt{"outward"} only when outer stability is an explicit requirement. Always annotate or at least describe the distortion in figure captions so readers do not mistake warped areas for standard projections.

\textbf{Simple tweaks.} If linework kinks, raise \texttt{squeeze\_factor} slightly (e.g., 0.45). If the focus
feels too tight, lower \texttt{zoom\_factor} toward 4--6. For reproducible comparisons, keep
\texttt{normalized\_center\ =\ TRUE} and reuse the same radii across runs.

\section{Exploring the implementation using Shiny}\label{exploring-the-implementation-using-shiny}

Explain the shiny app structure, and how it can be used interactively to explore a spatial data set.

Screenshot

\section{Application to Victorian ambulance transfer records}\label{application-to-victorian-ambulance-transfer-records}

We focus on one application: Victorian hospitals and residential aged care facilities (RACFs). Transfer counts are simulated; coordinates come from \texttt{conn\_fish}. We sample 10 hospitals and 10 RACFs to avoid clutter, use a grey basemap for contrast, and highlight how points, polygons, and connection lines behave under the same lens.

First, we can plot seperately the sampled hospitals and the age care facilities to see where they actually are.

\begin{figure}
\centering
\pandocbounded{\includegraphics[keepaspectratio]{paper-mapycusmaximus_files/figure-latex/hospitals-point-plot-1.pdf}}
\caption{\label{fig:hospitals-point-plot}Standard maps: hospital (red) and RACF (blue) points plotted over Victoria without fisheye.}
\end{figure}

Afterward, we can plot the connection between the hospital and the age care facilities to see the movement of the patient between the two as shown in the Figure \ref{fig:hospitals-basic-plot}.

\begin{figure}
\centering
\pandocbounded{\includegraphics[keepaspectratio]{paper-mapycusmaximus_files/figure-latex/hospitals-basic-plot-1.pdf}}
\caption{\label{fig:hospitals-basic-plot}Standard Victorian map with sampled hospitals (red), RACFs (blue), and simulated transfer lines on a grey background.}
\end{figure}

As we can see, comparing to the scale of Victoria, with the default scale method, the network mostly invisible. To magnify the network, we can apply a fisheye lens.

\begin{figure}
\centering
\pandocbounded{\includegraphics[keepaspectratio]{paper-mapycusmaximus_files/figure-latex/fisheye-plot-1.pdf}}
\caption{\label{fig:fisheye-plot}Fisheye view magnifies greater Melbourne while keeping statewide context; lines, points, and polygons stay aligned.}
\end{figure}

We could also include an interactive version of the plot above, thanks to the natively support and intergration of \texttt{ggplotly} with sf object.

The fisheye clarifies dense metro structure without losing state context. Hospitals and RACFs that previously overlapped separate cleanly; transfer lines remain connected to their endpoints because every layer uses the same centre and radii. Outside the glue zone, geometry remains stable, so readers can still place Melbourne within Victoria. To inspect a different hub (e.g., Geelong), change \texttt{center} to that polygon; multiple centres can be explored by re-running \texttt{sf\_fisheye()} with alternative centres and comparing panels side by side.

For polygons beyond metro Melbourne, the same lens settings apply; polygon edges bend smoothly but retain adjacency. Because transfer counts are simulated, readers should treat magnitude as illustrative only; spatial relationships (what becomes readable after zoom vs.~what remains hidden) follow from the geometry.

In this case, we can see that some point and connection was squished together since they were in the glue zone. We can change that by introducing a new center, and shrink the original focus zone. However, the mechanic of handling overlapping zone between multiple center is not implemented in this version yet.

\section{Discussion}\label{discussion}

\textbf{Contribution} mapycusmaximus provides an sf‑native implementation of the FGC fisheye that is projection‑aware, parameterised in normalised units, and safe across points, lines, and polygons. The package separates radial mapping from geometry orchestration, exposes explicit controls over focus, glue, and context, and preserves attributes and CRS invariants for reproducible pipelines with ggplot2.

\textbf{Relation to alternatives} Unlike cartograms (thematic distortion), hex/regular tile maps (discrete abstraction), or inset/multi‑panel layouts (spatial separation), the FGC lens delivers continuous magnification within a single map while preserving topology and bearings. This reduces cognitive load for readers who must relate local phenomena to their broader geography.

\textbf{Limitations} The fisheye introduces non‑metric distortion in the focus and glue; therefore, use it for visual exploration and communication, not for metric analysis. Aggressive \texttt{zoom} or \texttt{squeeze} can impair legibility near the glue boundary; conservative defaults and \texttt{revolution\ =\ 0} are recommended for publication maps. When comparing multiple regions, prefer \texttt{normalized\_center\ =\ TRUE} with fixed radii to ensure visual comparability. At present, exact matching of focus and glue radii across separately transformed layers may require a manual step (the user have to manually merge the two or more layers, perform the fisheye transformation, then seperated the transformed layers).

\textbf{Future work} Planned extensions include anisotropic or elliptical profiles, multi‑focus blending, first‑class raster support via warped grids and resampling, and interactive focus selection for exploratory analysis. We also plan an API for shared normalisation and radius locking across layers (e.g., a \texttt{combine\_fisheye}) so that multiple layers can be warped with identical scale and then returned transformed. Performance improvements via vectorised geometry walkers or GPU acceleration would benefit dense polygonal datasets. Clear figure captions and scale disclaimers remain essential to communicate the presence and intent of distortion.

\section{Conclusion}\label{conclusion}

FGC fisheye transformations offer a concise, CRS‑aware way to emphasise local structure without losing geographic context. By starting from a point‑wise radial map and integrating carefully with sf for geometry reconstruction, the approach keeps figures continuous and overlays aligned. The examples demonstrate clearer narratives for metropolitan focus while maintaining state‑ or nation‑level context.

\section{AI Use Declaration}\label{ai-use-declaration}

We used AI tools to assist with code refactoring and drafting portions of the text. All methods, parameter settings, and claims were designed and reviewed by the authors, and we verified outputs with the package's test suite and example renders.

\section{Resources}\label{resources}

The github link for this paper is \href{https://github.com/Alex-Nguyen-VN/paper-mapycusmaximus}{here}.

The mapycusmaximus package is available on \href{https://github.com/Alex-Nguyen-VN/mapycusmaximus}{GitHub}.

The slideshow for this package can be found \href{https://github.com/Alex-Nguyen-VN/Talk/tree/main/mapycus-slide}{here}.

\bibliography{paper-mapycusmaximus.bib}

\address{%
Thanh Cuong Nguyen\\
Monash University\\%
Department of Econometrics and Business Statistics\\ Melbourne, Australia\\
%
\url{https://alex-nguyen-vn.github.io}\\%
\textit{ORCiD: \href{https://orcid.org/0000-0000-0000-0000}{0000-0000-0000-0000}}\\%
\href{mailto:thanhcuong10091992@gmail.com}{\nolinkurl{thanhcuong10091992@gmail.com}}%
}

\address{%
Michael Lydeamore\\
Monash University\\%
Department of Econometrics and Business Statistics\\ Melbourne, Australia\\
%
\url{https://www.michaellydeamore.com}\\%
\textit{ORCiD: \href{https://orcid.org/0000-0001-6515-827X}{0000-0001-6515-827X}}\\%
\href{mailto:michael.lydeamore@monash.edu}{\nolinkurl{michael.lydeamore@monash.edu}}%
}

\address{%
Dianne Cook\\
Monash University\\%
Department of Econometrics and Business Statistics\\ Melbourne, Australia\\
%
\url{https://www.dicook.org}\\%
\textit{ORCiD: \href{https://orcid.org/0000-0002-3813-7155}{0000-0002-3813-7155}}\\%
\href{mailto:dicook@monash.edu}{\nolinkurl{dicook@monash.edu}}%
}
