% !TeX root = RJwrapper.tex
\title{Focus-Glue-Context Fisheye Transformations for Spatial Visualization}


\author{by Thanh Cuong Nguyen, Michael Lydeamore, and Dianne Cook}

\maketitle

\abstract{%
Fisheye views magnify local detail while preserving context, yet projection-aware, scriptable tools for R spatial analysis remain limited. mapycusmaximus introduces a Focus--Glue--Context (FGC) fisheye transform for numeric coordinates and sf geometries. Acting radially around a chosen center, the transform defines a magnified focus (r\_in), a smooth transitional glue zone (r\_out), and a fixed exterior. Distances expand or compress via a zoom factor and power-law squeeze, with an optional angular twist enhancing continuity. The method is projection-conscious: lon/lat inputs are reprojected to suitable CRSs (e.g., GDA2020/MGA55), normalized for stable parameter control, and restored afterward. A geometry-safe engine (st\_transform\_custom) supports all feature types, maintaining ring closure and metadata. The high-level sf\_fisheye() integrates with tidyverse, ggplot2, and Shiny, with built-in datasets and tests ensuring reproducibility. By coupling coherent radial warps with tidy, CRS-aware workflows, mapycusmaximus enables spatial exploration that emphasizes local structure without losing global context.
}

\section{Introduction}\label{introduction}

Maps that reveal fine local structure without losing broader context face a persistent challenge: zooming in hides regional patterns, while small-scale views suppress local detail. Traditional solutions---insets, multi-panel displays, aggressive generalization---break spatial continuity and increase cognitive load. What if we could smoothly magnify a metropolitan core \emph{while keeping it embedded} in its state-level context?

This package implements a Focus--Glue--Context (FGC) fisheye transformation that continuously warps geographic space: a chosen focus region magnifies, surrounding areas compress into a ``glue'' transition zone, and outer context remains stable. Unlike discrete zoom levels or disconnected insets, the transformation operates directly on vector geometry coordinates, preserving topology and enabling reproducible, pipeline-friendly cartography within R's sf and ggplot2 ecosystem.

The intellectual lineage of focus+context visualization traces back to \citet{furnas1986}'s \emph{degree-of-interest} function, which formalized how to prioritize salient regions while retaining global structure. \citet{sarkar1992} and \citet{sarkar1994} extended this to geometric distortion, demonstrating smooth magnification transitions for graph visualization. Subsequent innovations explored diverse lenses: hyperbolic geometry for hierarchies \citep{lamping1995}, distortion-view frameworks \citep{carpendale2001}, and ``magic lens'' overlays \citep{bier1993}. By 2008, \citet{cockburn2008}'s comprehensive review synthesized two decades of research across overview+detail, zooming, and focus+context paradigms.

In cartography, the need for nonlinear magnification emerged independently. \citet{snyder1987} developed ``magnifying-glass'' azimuthal projections with variable radial scales---mathematical foundations still cited today. \citet{harrie2002} created variable-scale functions for mobile devices where user position appears large-scale against small-scale surroundings. The crucial breakthrough came from \citet{yamamoto2009} and \citet{yamamoto2012}: their \textbf{Focus+Glue+Context model} introduced an intermediate ``glue'' region that absorbs distortion, preventing the excessively warped roads and boundaries that plagued earlier fisheye maps. This three-zone architecture proved particularly effective for pedestrian navigation and mobile web services.

Parallel developments in statistical graphics tackled the ``crowding problem''---high-dimensional data collapsing into projection centers. \citet{JMLR:v9:vandermaaten08a}'s t-SNE uses heavy-tailed distributions to spread points, while \citet{mcinnes2020umapuniformmanifoldapproximation}'s UMAP leverages topological methods. Most relevant to our geometric approach: \citet{laa2020} applies \emph{radial transformations} to tour projections, maintaining the interpretability of linear methods while mitigating overplotting. Implemented in R's tourr package, it demonstrates how well-designed radial warps can reveal structure without the distortions of fully nonlinear embeddings.

Within R's spatial ecosystem, sf \citep{RJ-2018-009} provides robust vector handling and CRS transformations, while ggplot2 \citep{wickham2016} offers declarative visualization grammar. Yet a gap remained: existing tools addressed \emph{related} distortion needs but not continuous geometric fisheye lenses. This package fills that niche by formalizing an sf-native FGC radial model with controllable zone parameters, optional angular effects, automatic normalization, and safe geometry handling across points, lines, and polygons.

\section{Background: Alternative Approaches to the Detail-Context Problem}\label{background-alternative-approaches-to-the-detail-context-problem}

Before diving into fisheye mechanics, it's worth understanding how R's spatial ecosystem currently handles the detail-versus-context tradeoff---and why those solutions, while valuable, leave room for continuous lens-based warping.

\textbf{Cartograms: Thematic distortion.} The cartogram family \citep{gastner2004} intentionally distorts geographic areas to encode variables---population density reshapes regions so area becomes proportional to demographic weight.

\pandocbounded{\includegraphics[keepaspectratio]{paper-mapycusmaximus_files/figure-latex/plot-cart-1.pdf}}

This fundamentally differs from focus+context: cartograms \emph{substitute} spatial accuracy for data encoding, often severely disrupting shapes and adjacencies. A population cartogram makes California balloon while Wyoming shrinks, trading geographic fidelity for thematic insight. FGC fisheye, conversely, preserves relative positions and topology while magnifying a \emph{chosen} spatial region, not a data-driven variable. The use cases diverge: cartograms answer ``how does this variable dominate space?'' while fisheye lenses answer ``what local detail exists within this broader geography?''

\textbf{Hexagon tile maps: Discrete abstraction.} Packages like \texttt{geogrid} and visualizations using \texttt{sf::st\_make\_grid()} replace irregular polygons with regular hexagonal or square tiles, each representing an administrative unit.

\pandocbounded{\includegraphics[keepaspectratio]{paper-mapycusmaximus_files/figure-latex/geo-grid-plot-1.pdf}}

As seen in the plot above, tile maps \emph{abstracts away} precise geography entirely, treating space as a topology-preserving tessellation where ``neighbors touch'' matters more than accurate boundaries. Tile maps excel at avoiding size bias (Mildura gets equal visual weight to Yarra) and creating aesthetic, clutter-free layouts. However, they abandon continuous spatial relationships: you cannot identify precise locations, measure distances, or overlay point data meaningfully. Hexbin aggregation for point data (via \texttt{ggplot2::geom\_hex()}) serves a different purpose---density estimation---rather than focus+context navigation.

\textbf{Multi-panel approaches: Spatial separation.} Tools like \texttt{cowplot::ggdraw()}\citep{cowplot} create side-by-side views: one panel shows overview, another shows zoomed detail.

\pandocbounded{\includegraphics[keepaspectratio]{paper-mapycusmaximus_files/figure-latex/cow-plot-plot-1.pdf}}

These are effective for static reports but require viewers to mentally integrate separate views, and they don't preserve the \emph{embedded} relationship between focus and context within a single continuous geography. Futhermore, if you introduce one or more elements into the plot like filling value equal to a variable, the audience will have a hard time identify the zoomed detail.

\textbf{Why FGC fisheye offers something distinct.} None of these approaches provide \emph{continuous geometric magnification within a single, topology-preserving map}. Cartograms distort for data, not user-chosen focus. Tile maps abstract away geography. Multi-panel tools spatially separate context. The fisheye lens keeps everything in one frame---roads bend smoothly, metropolitan detail enlarges, but you still see how the city sits within its state. It's a geometric \emph{warp} rather than a data-driven \emph{substitution} or panel-based \emph{separation}. This matters for use cases like: examining hospital networks in Melbourne while maintaining Victorian context, exploring census tracts in a metro core without losing county boundaries, or analyzing transit lines with their regional hinterland visible.

With this landscape established, we now turn to the technical implementation: how does the Focus--Glue--Context transformation actually work, and how does this package make it accessible within R's spatial workflows?

\section{Focus--Glue--Context Transformation}\label{focusgluecontext-transformation}

HERE YOU EXPLAIN THE ALGORITHM AND INCLUDE SOME CODE FROM PACKAGE THAT DOES THE PARTS. USE A SIMPLE EXAMPLE LIKE THE RECTANGLE OF DOTS TO EXPLAIN

\subsection{Algorithm}\label{algorithm}

\subsection{Parameters}\label{parameters}

\subsection{Common choices}\label{common-choices}

\section{Examples of use}\label{examples-of-use}

SHOW THE WAYS THAT IT CAN BE USED FOR THE VICTORIAN AMBULANCE DATA: Just the map with hospital locations, map with transfers, map with convex hulls, map with two focal points, then maybe a raster map

\section{Discussion}\label{discussion}

HERE YOU SUMMARISE WHAT THE PAPER CONTRIBUTED IN ONE PARAGRAPH AND SUGGEST NEW WORK THAT MIGHT BE DONE THAT YOU DIDN'T HAVE TIME TO DO

\bibliography{paper-mapycusmaximus.bib}

\address{%
Thanh Cuong Nguyen\\
Monash University\\%
Department of Econometrics and Business Statistics\\ Melbourne, Australia\\
%
\url{https://alex-nguyen-vn.github.io}\\%
\textit{ORCiD: \href{https://orcid.org/0000-0000-0000-0000}{0000-0000-0000-0000}}\\%
\href{mailto:thanhcuong10091992@gmail.com}{\nolinkurl{thanhcuong10091992@gmail.com}}%
}

\address{%
Michael Lydeamore\\
Monash University\\%
Department of Econometrics and Business Statistics\\ Melbourne, Australia\\
%
\url{https://www.michaellydeamore.com}\\%
\textit{ORCiD: \href{https://orcid.org/0000-0001-6515-827X}{0000-0001-6515-827X}}\\%
\href{mailto:michael.lydeamore@monash.edu}{\nolinkurl{michael.lydeamore@monash.edu}}%
}

\address{%
Dianne Cook\\
Monash University\\%
Department of Econometrics and Business Statistics\\ Melbourne, Australia\\
%
\url{https://www.dicook.org}\\%
\textit{ORCiD: \href{https://orcid.org/0000-0002-3813-7155}{0000-0002-3813-7155}}\\%
\href{mailto:dicook@monash.edu}{\nolinkurl{dicook@monash.edu}}%
}
